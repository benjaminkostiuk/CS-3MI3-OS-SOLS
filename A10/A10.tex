\documentclass[12pt, fleqn]{article}

\usepackage[utf8]{inputenc}
\usepackage{graphicx}
\usepackage{paralist}
\usepackage{booktabs}
\usepackage{hyperref}
\usepackage{amsmath}
\usepackage{amsfonts}
\usepackage{amssymb}
\usepackage{amsthm}
\usepackage{color}
\usepackage[nounderscore]{syntax}
\usepackage{mdwtab}
\usepackage{mathpartir}

% \renewcommand{\texttt}[1]{\OldTexttt{\color{teal}{#1}}}
\newcommand{\pnote}[1]{{\langle \text{#1} \rangle}}
\newcommand{\mname}[1]{\mbox{\sf #1}}
\definecolor{mGreen}{rgb}{0,0.6,0}
\definecolor{mGray}{rgb}{0.5,0.5,0.5}
\definecolor{mPurple}{rgb}{0.58,0,0.05}
\definecolor{mGreen2}{rgb}{0.05,0.65,0.05}
\definecolor{mGray2}{rgb}{0.55,0.55,0.55}
\definecolor{mPurple2}{rgb}{0.63,0.05,0.05}
\definecolor{backgroundColour}{rgb}{0.95,0.95,0.92}
\definecolor{backgroundColour2}{rgb}{0.95,0.92,0.95}

\definecolor{t_comment}{rgb}{0.2,1,0.2}
\definecolor{t_mGray}{rgb}{0.5,0.5,0.5}
\definecolor{t_mPurple}{rgb}{0.58,0,0.05}
\definecolor{t_blue}{rgb}{0.4,0.6,0.8}
\definecolor{t_mGreen2}{rgb}{0.05,0.65,0.05}
\definecolor{t_mGray2}{rgb}{0.75,0.75,0.75}
\definecolor{t_mPurple2}{rgb}{0.63,0.05,0.05}
\definecolor{t_bg}{rgb}{0.15,0.15,0.18}

\setlength{\parindent}{0pt}
\setlength {\topmargin} {-.15in}
\oddsidemargin 0mm
\evensidemargin 0mm
\textwidth 160mm
\textheight 200mm

\pagestyle {plain}
\pagenumbering{arabic}

\newcounter{stepnum}

\title{3MI3 Assignment 10 Solution}
\author{x}
\date{\today}

\begin{document}


\begin{center}

    {\large \textbf{COMPSCI 3MI3}}\\[8mm]
    {\huge \textbf{Assignment 10}}\\[6mm]
  
\end{center}

\medskip

\section{Solution Set}

\subsection{Q1}
We assume the syntax and semantics for our extended simply-typed $\lambda$-Calculus defined in the slides for topic 11. We propose the following extension to the syntax and semantics for pattern-matching and \textsl{match}.

\begin{grammar}
    <t> ::= \dots

    <v> ::= \dots

    <p> ::= \dots
        \alt nil[<T>]
        \alt cons[<T>] <p> <p>
\end{grammar}
We extend \textsl{match} with the following matching rules:
\begin{equation}
    { \textsl{match}(\textsl{nil}[T], \textsl{nil}[T]) = [x \mapsto x]} \tag{M-Nil}
\end{equation}
\begin{equation}
    \inferrule{ \textsl{match}(p_1, v_1) = \sigma_1 \and \textsl{match}(p_2, v_2) = \sigma_2  }{ \textsl{match}(\textsl{cons}[T]\:p_1\:p_2, \textsl{cons}[T]\:v_1\:v_2) = \sigma_1 \circ \sigma_2} \tag{M-Cons}
\end{equation}
Where $T$ is a type and $\circ$ is function composition.

\subsection{Q2}

We assume the syntax and semantics for our simply-typed $\lambda$-Calculus extended with lists defined in the slides for topic 11. We define the syntax for our new terms as follows:

\begin{grammar}
    <t> ::= \dots
    \alt index[<T>] <t> <t>
    \alt length[<T>] <t>
    \alt concat[<T>] <t> <t>
    \alt filter[<T>] <t> <t>

    <v> ::= \dots

    <T> ::= \dots
\end{grammar}

We define the semantics of each of our new terms below.

\begin{enumerate}[(a)]
    \item List indexing
    We define the following evaluation and typing semantics for list indexing:

    \begin{equation}
        { \textsl{index}[S]\:(\textsl{cons}[T]\:v_1\:v_2)\:0 \rightarrow v_1} \tag{E-IndexZero}
    \end{equation}
    \begin{equation}
        { \textsl{index}[S]\:(\textsl{cons}[T]\:v_1\:v_2)\:(\textsl{succ}\:nv) \rightarrow \textsl{index}[S]\:v_2\:nv} \tag{E-IndexSucc}
    \end{equation}
    \begin{equation}
        \inferrule{ t_1 \rightarrow t_1' }{ \textsl{index}[T]\:t_1\:t_2 \rightarrow \textsl{index}[T]\:t_1'\:t_2} \tag{E-Index1}
    \end{equation}
    \begin{equation}
        \inferrule{ t_2 \rightarrow t_2' }{ \textsl{index}[T]\:v_1\:t_2 \rightarrow \textsl{index}[T]\:v_1\:t_2'} \tag{E-Index2}
    \end{equation}
    \begin{equation}
        \inferrule{ \Gamma \vdash t_1 : \textsl{List T} \and \Gamma \vdash t_2 : \textsl{Nat} }{ \Gamma \vdash \textsl{index}[T]\:t_1\:t_2 : T} \tag{T-Index}
    \end{equation}

    \item Length\\
    We define the following evaluation and typing semantics for length:

    \begin{equation}
        { \textsl{length}[S](\textsl{nil}[T]) \rightarrow 0 } \tag{E-LengthNil}
    \end{equation}
    \begin{equation}
        { \textsl{length}[S](\textsl{cons}[T] \: v_1 \: v_2) \rightarrow  \textsl{succ}\:(\textsl{length}[S]\:v_2) } \tag{E-LengthCons}
    \end{equation}
    \begin{equation}
        \inferrule{ t \rightarrow t' }{ \textsl{length}[T]\:t \rightarrow \textsl{length}[T]\:t' }\tag{E-Length}
    \end{equation}
    \begin{equation}
        \inferrule{ \Gamma \vdash t : \textsl{List T} }{ \Gamma \vdash \textsl{length}[T]\:t : \textsl{Nat}}\tag{T-Length}
    \end{equation}
    
    \item Concatenation\\
    We define the following evaluation and typing semantics for concatenation:

    \begin{equation}
        { \textsl{concat}[S]\:(\textsl{nil}[T])\:v_2 \rightarrow v_2 } \tag{E-ConcatNil} 
    \end{equation}
    \begin{equation}
        { \textsl{concat}[S]\:(\textsl{cons}[T]\:v_1\:v_2)\:v_3 \rightarrow \textsl{cons}[T]\:v_1\:(\textsl{concat}[S]\:v_2\:v_3) } \tag{E-ConcatCons} 
    \end{equation}
    \begin{equation}
        \inferrule{ t_1 \rightarrow t_1' }{ \textsl{concat}[T]\:t_1\:t_2 \rightarrow \textsl{concat}[T]\:t_1'\:t_2 } \tag{E-Concat1} 
    \end{equation}
    \begin{equation}
        \inferrule{ t_2 \rightarrow t_2' }{ \textsl{concat}[T]\:v_1\:t_2 \rightarrow \textsl{concat}[T]\:v_1\:t_2' } \tag{E-Concat2 } 
    \end{equation}
    \begin{equation}
        \inferrule{ \Gamma \vdash t_1 : \textsl{List T} \and \Gamma \vdash t_2 : \textsl{List T} }
        { \Gamma \vdash \textsl{concat}[T]\:t_1\:t_2 : \textsl{List T} } \tag{T-Concat} 
    \end{equation}

    \item Filtering\\
    We define the following evaluation and typing semantics for filtering:

    \begin{equation}
        { \textsl{filter}[S]\:v_1\:(\textsl{nil}[T]) \rightarrow \textsl{nil}[T]} \tag{E-FilterNil}
    \end{equation}
    \begin{equation}
        { \textsl{filter}[S]\:v_1\:(\textsl{cons}[T]\:v_2\:v_3) \rightarrow \texttt{if}\:v_1\:v_2\:\texttt{then}\: \textsl{cons}[T]\:v_2\:(\textsl{filter}[S]\:v_1\:v_3)\:\texttt{else}\:\textsl{filter}[S]\:v_1\:v_3} \tag{E-FilterCons}
    \end{equation}
    % \begin{equation}
    %     \inferrule{ v_1\:v_2 \rightarrow^* \textsl{true} }
    %     { \textsl{filter}[S]\:v_1\:(\textsl{cons}[T]\:v_2\:v_3) \rightarrow \textsl{cons}[T]\:v_2\:(\textsl{filter}[S]\:v_1\:v_3)} \tag{E-FilterCons1}
    % \end{equation}
    % \begin{equation}
    %     \inferrule{ v_1\:v_2 \rightarrow^* \textsl{false} }
    %     { \textsl{filter}[S]\:v_1\:(\textsl{cons}[T]\:v_2\:v_3) \rightarrow \textsl{filter}[S]\:v_1\:v_3} \tag{E-FilterCons2}
    % \end{equation}
    \begin{equation}
        \inferrule{ t_1 \rightarrow t_1' }
        { \textsl{filter}[T]\:t_1\:t_2 \rightarrow \textsl{filter}[T]\:t_1'\:t_2} \tag{E-Filter1}
    \end{equation}
    \begin{equation}
        \inferrule{ t_2 \rightarrow t_2' }
        { \textsl{filter}[T]\:v_1\:t_2 \rightarrow \textsl{filter}[T]\:v_1\:t_2'} \tag{E-Filter2}
    \end{equation}
    \begin{equation}
        \inferrule{\Gamma \vdash t_1 : T \Rightarrow \textsl{Bool} \and \Gamma \vdash t_2 : \textsl{List T} }
        { \Gamma \vdash \textsl{filter}[T]\:t_1\:t_2 : \textsl{List T} } \tag{T-Filter} 
    \end{equation}


\end{enumerate}

\end {document}