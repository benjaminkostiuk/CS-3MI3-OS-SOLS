\documentclass[12pt, fleqn]{article}

\usepackage{graphicx}
\usepackage{paralist}
\usepackage{listings}
\usepackage{booktabs}
\usepackage{hyperref}
\usepackage{amsmath}
\usepackage{amssymb}
\usepackage{amsthm}
\usepackage{color}

\renewcommand{\texttt}[1]{\OldTexttt{\color{teal}{#1}}}
\newcommand{\pnote}[1]{{\langle \text{#1} \rangle}}
\newcommand{\mname}[1]{\mbox{\sf #1}}
\definecolor{mGreen}{rgb}{0,0.6,0}
\definecolor{mGray}{rgb}{0.5,0.5,0.5}
\definecolor{mPurple}{rgb}{0.58,0,0.05}
\definecolor{mGreen2}{rgb}{0.05,0.65,0.05}
\definecolor{mGray2}{rgb}{0.55,0.55,0.55}
\definecolor{mPurple2}{rgb}{0.63,0.05,0.05}
\definecolor{backgroundColour}{rgb}{0.95,0.95,0.92}
\definecolor{backgroundColour2}{rgb}{0.95,0.92,0.95}

\definecolor{t_comment}{rgb}{0.2,1,0.2}
\definecolor{t_mGray}{rgb}{0.5,0.5,0.5}
\definecolor{t_mPurple}{rgb}{0.58,0,0.05}
\definecolor{t_blue}{rgb}{0.4,0.6,0.8}
\definecolor{t_mGreen2}{rgb}{0.05,0.65,0.05}
\definecolor{t_mGray2}{rgb}{0.75,0.75,0.75}
\definecolor{t_mPurple2}{rgb}{0.63,0.05,0.05}
\definecolor{t_bg}{rgb}{0.15,0.15,0.18}

\setlength{\parindent}{0pt}
\setlength {\topmargin} {-.15in}
\oddsidemargin 0mm
\evensidemargin 0mm
\textwidth 160mm
\textheight 200mm

\pagestyle {plain}
\pagenumbering{arabic}

\newcounter{stepnum}

\title{3MI3 Assignment 1 Solution}
\author{x}
\date{\today}

\begin{document}

\begin{center}

    {\large \textbf{COMPSCI 3MI3}}\\[8mm]
    {\huge \textbf{Assignment 1}}\\[6mm]
    {\large \today}
  
\end{center}

\medskip

\section{Solution Set}

\subsection{Q1}

Define the relation $R'$ as follows:
$$R' = R \cup \{(s,s)~|~s \in S\}$$

\noindent
To show $R'$ is the \emph{reflexive closure} of $R$ we must show $(1)$ $R \subseteq R'$, $(2)$ $R'$ 
is reflexive and $(3)$ $R'$ is the smallest reflexive relation such that $R \subseteq R'$.

\medskip

$(1)$ Prove $R \subseteq R'$
\begin{align*}
    &\phantom{{}=} R' & \\
    &= R \cup \{ (s, s)~|~s \in S\} & \pnote{def. of $R'$} \\
    & \supseteq R & \pnote{$A \subseteq A \cup B$ }
\end{align*}

$(2)$ Show $R'$ is reflexive.\\
Since the relation $R'$ contains the set of all reflexive pairs in $S$
$\{ (s, s) ~|~ s \in S \}$ we know that $R'$ is reflexive by definition. 

\medskip

$(3)$ Show $R'$ is the smallest reflexive relation possible such that $R \subseteq R'$.\\
We must show $R'$ is the smallest possible set that satisfies the above two properties. 
Suppose $K$ is a reflexive relation on $S$ extending $R$. By reflexivity of $K$, we know $\{ (s, s) \in S \} \subseteq K$. 
Since we also know $K$ extends $R$ then $R \subseteq K$, it follows that $R \cup \{ (s, s) \in S \} \subseteq K$, which is $R' \subseteq K$. Thus, $R'$ must be the smallest possible reflexive relation extending $R$.

\medskip 
We have shown that $R'$ is the reflexive closure of $R$. 

\subsection{Q2}

We suppose that $R$ is some binary relation on $S$ and $P$ is a predicate on $S$ that preserves $R$. 
Let $P^*$ be the \emph{relexive transitive} closure of $R$. We will show that $P$ also preserves $P^*$.

\medskip

If $P$ preserves $P^*$ then for any $s, t \in S$ the following should hold:

$$P(s) \land s~P^*~t \rightarrow P(t)$$

This is equivalent to saying that for all pairs $(s, t) \in P^*$ if $P(s)$ holds then $P(t)$ holds. We can then
define $P^*$ as:
$$P^* = R \cup K \cup T$$
where $K$ contains all reflexive pairs $(s, s)$ in $S$ and $T$ contains all pairs that together make $P^*$ transitive.
Thus, to prove $P$ preserves $P^*$ we show that $P$ preserves $R, K, T$ such that for any pair $(s, t) \in R \cup K \cup T$ we
have $P(s) \rightarrow P(t)$.

\begin{enumerate}
    \item[(1)] ($P$ preserves $R$) This is trivial since we know that $P$ preserves  $R$ as given in the question, thus for any
    pair $(s, t) \in R$ we know $P(s) \rightarrow P(t)$.
     
    \item[(2)] ($P$ preserves $K$) Since all pairs in $K$ are reflexive they all share the form $(s, s)$ where $s \in S$. 
    Thus for any $(s, s) \in K$ we have $P(s) \rightarrow P(s)$, which is true by reflexivity of implication. Therefore, $P$
    preserves $K$.

    \item[(3)] ($P$ preserves $T$) Since $T$ contains all pairs that together make $P^*$ transitive we know that for any pair
    $(s, t) \in T$ there must exist two pairs $(s, r)$,$(r, t) \in R \cup T$. 
    We can prove by induction that for any pair $(s, t) \in T$ we have $P(s) \rightarrow P(t)$. For any pair $(s, t) \in T$
    if $(s, r),(r,t) \in R$ then the pair can be obtained in ``one step of transitivity" and from (1) we know such 
    that $P(s) \rightarrow P(r)$ and $P(r) \rightarrow P(t)$. By transitivity of implication we have that $P(s) \rightarrow P(t)$.
    Similarly, all other pairs $(s, t) \in T$ follow this principle since they are formed in ``one step of transitvity'' from two pairs 
    $(s, r)$ and $(r, t)$ that already follow the principle. Thus $P$ preserves $T$.
\end{enumerate}

Therefore, we have shown that $P$ preserves $P^*$.

\subsection{Q3}

Define the relation $R^+$ as given in the question:
$$R_0 = R$$ 
$$R_{i+1} = R_i \cup \{(s,u) \: | \: \text{for some} \: t, (s,t) \in R_i \text{ and } (t,u) \in R_i\}$$
$$R^{+} = \bigcup_i R_i$$

To show $R^+$ is the \emph{transitive closure} of $R$ we must show $(1)$ $R \subseteq R^+$, $(2)$ $R^+$ is transitive,
$(3)$ $R^+$ is the smallest transitive relation such that $R \subseteq R^+$.

\medskip

$(1)$ Prove $R \subseteq R^+$
\begin{align*}
&\phantom{{}=} R+ & \\
&= \bigcup_i R_i & \pnote{def. of $R^+$} \\
&\supseteq R_0 & \pnote{def. of $\bigcup$, $A \subseteq A \cup B$} \\
&= R  & \pnote{$R_0 = R$} \\
\end{align*}

$(2)$ Show $R^+$ is transitive.\\
Suppose some $a, b, c \in S$ with $a~R_i~b$ and $b~R_i~c$ for some $i$. By definition of $R^+$ 
the pair $(a, c)$ will appear in $R_{i+1}$ such that $a~R_{i+1}~c$. Since $R^+$ is the union of all $R_i$
the pairs $(a, b)$,$(b, c)$,$(a, c)$ $\in R^+$, thus $R^+$ is transitive by definition.

\medskip

$(3)$ Show $R^+$ is the smallest transitive relation possible such that $R \subseteq R^+$.\\
We must show $R^+$ is the smallest possible set that satisfies the above two properties. Suppose $T$ is a transitive relation on $S$ that
extends $R$. We know $T$ extends $R$ so $R \subseteq T$. Additionally, since $T$ is transitive
for any $a, b, c \in S$ such that $a~R~b$ and $b~R~c$ we know $a~T~c$, which implies $R_1 \subseteq T$. We can then inductively prove
that for any $a, b, c \in S$ with $a~R_i~b$ and $b~R_i~c$ we will have $a~T~c$ thus $R_{i+1} \subseteq T$, or simply $R_i \subseteq T$ for all $i$.
It then follows that $\bigcup_i R_i \subseteq T$ which is $R^+ \subseteq T$. Therefore, $R^+$ must be the smallest transitive relation extending $R$.

\subsection{Q4}
Define the function $\mname{fib} : \mathbb{N} \rightarrow \mathbb{N}$ such that $\mname{fib}(n)$ returns the
n\textsuperscript{th} number of the Fibbonaci sequence (0-indexed). $\mname{fib}$ is defined by:
\[\mname{fib}(n) = 
    \left\{\begin{array}{ll}
            0 & \textrm{if } n = 0 \\
            1 & \textrm{if } n = 1 \\
            \mname{fib}(n-1) + \mname{fib}(n-2) & \textrm{if } n \ge 2
            \end{array}
    \right.\]

We will prove that each element in the Fibbonaci sequence above the 2\textsuperscript{nd} is
greater than the preceeding number. Formally, we prove that $P(n) \equiv \mname{fib}(n) > \mname{fib}(n - 1)$
for all $n \in \mathbb{N}$ with $n > 2$ by ordinary induction on $n$.

\medskip

\emph{Base case: n = 3}. Show $P(3)$ holds.
\begin{align*}
    & \mname{fib}(3) & \pnote{RHS} \\
    &= \mname{fib}(3 - 1) + \mname{fib}(3 - 2) &  \pnote{def. of $\mname{fib}.3$} \\
    &= \mname{fib}(2) + \mname{fib}(1) & \pnote{arithmetic} \\
    &= \mname{fib}(2) + 1 & \pnote{def. of $\mname{fib}.2$} \\
    &> \mname{fib}(2) & \pnote{$b > 0 \rightarrow a + b > a$; LHS}
\end{align*}
So $P(3)$ holds.

\medskip

\emph{Induction step: n $>$ 3}. Assume $P(n)$ holds. We must show $P(n+1)$ holds.
\begin{align*}
    & \mname{fib}(n + 1) & \pnote{RHS} \\
    &= \mname{fib}(n + 1 - 1) + \mname{fib}(n + 1 - 2) &  \pnote{def. of $\mname{fib}.3$} \\
    &= \mname{fib}(n) + \mname{fib}(n - 1) & \pnote{arithmetic} \\
    &> \mname{fib}(n) & \pnote{$b > 0 \rightarrow a + b > a$} \\
    &= \mname{fib}(n + 1 - 1) & \pnote{arithmetic; LHS}
\end{align*}
Thus, $P(n+1)$ holds.

\medskip

Therefore, $P(n)$ holds for all $n \in \mathbb{N}$ by ordinary induction. We have shown that each element in the 
Fibbonaci sequence above the 2\textsuperscript{nd} is greater than the preceeding number.

\subsection{Q5}

Using the same definition of $\mname{fib}$ from Q4 we will prove that each element in the Fibbonaci sequence above the 2\textsuperscript{nd} is
greater than the preceeding number. Formally, we prove that $P(n) \equiv \mname{fib}(n) > \mname{fib}(n - 1)$
for all $n \in \mathbb{N}$ with $n > 2$ by complete induction on $n$.

\emph{Base case: n = 3}. Show $P(3)$ holds.
\begin{align*}
    & \mname{fib}(3) & \pnote{RHS} \\
    &= \mname{fib}(3 - 1) + \mname{fib}(3 - 2) &  \pnote{def. of $\mname{fib}.3$} \\
    &= \mname{fib}(2) + \mname{fib}(1) & \pnote{arithmetic} \\
    &= \mname{fib}(2) + 1 & \pnote{def. of $\mname{fib}.2$} \\
    &> \mname{fib}(2) & \pnote{$b > 0 \rightarrow a + b > a$; LHS}
\end{align*}
So $P(3)$ holds.

\medskip

\emph{Base case: n = 4}. Show $P(4)$ holds.
\begin{align*}
    & \mname{fib}(4) & \pnote{RHS} \\
    &= \mname{fib}(4 - 1) + \mname{fib}(4 - 2) &  \pnote{def. of $\mname{fib}.3$} \\
    &= \mname{fib}(3) + \mname{fib}(2) & \pnote{arithmetic} \\
    &= \mname{fib}(3) + \mname{fib}(1) + \mname{fib}(0) & \pnote{def. of $\mname{fib}$; arithmetic} \\
    &= \mname{fib}(3) + 1 & \pnote{def. of $\mname{fib}$; arithmetic} \\
    &> \mname{fib}(3) & \pnote{$b > 0 \rightarrow a + b > a$; LHS}
\end{align*}
So $P(4)$ holds.

\medskip

\emph{Induction step: n $>$ 4}. Assume $P(m)$ holds for all $2 < m < n$. We must show $P(n)$ holds.
\begin{align*}
    &\phantom{{}=} \mname{fib}(n) & \pnote{RHS} \\
    &= \mname{fib}(n - 1) + \mname{fib}(n - 2) &  \pnote{def. of $\mname{fib}$} \\
    &> \mname{fib}(n - 2) + \mname{fib}(n - 2) & \pnote{Ind. hyp. $P(n - 1)$; Monotonicity of +} \\
    &> \mname{fib}(n - 2) + \mname{fib}(n - 3) & \pnote{Ind. hyp. $P(n - 2)$; Monotonicity of +} \\
    &= \mname{fib}(n - 1) & \pnote{def. of \mname{fib}; LHS}
\end{align*}
Thus $P(n)$ holds for $n > 4$.

\medskip
Therefore, $P(n)$ holds for all $n \in \mathbb{N}$ by complete induction. We have shown that each element in the 
Fibbonaci sequence above the 2\textsuperscript{nd} is greater than the preceeding number.

\subsection{Q6}

Define a binary tree data structure $\mname{BinTree}$ representative of binary search trees over an
arbitrary data type $\mname{a}$ for which the properties of binary search trees are respected.
$\mname{BinTree}$ is defined by pattern matching with the following constructors:
\begin{itemize}
    \item[] $\mname{Leaf}: \mname{a} \rightarrow \mname{BinTree}$
    \item[] $\mname{Fork}: \mname{BinTree} \times \mname{a} \times \mname{BinTree} \rightarrow \mname{BinTree}$
\end{itemize}
We define $P(t) \equiv$ search operations over the binary search tree $t$ modeled as a $\mname{BinTree}$ only need to search one branch of $t$.
We will prove $P(t)$ for all $t \in \mname{BinTree}$ by structural induction on $t$.

\medskip
\emph{Base case: t = $\mname{Leaf}(n)$}. Prove $P(t)$ holds.

\medskip
It is trivially obvious that the search algorithm over $t$ can only check one element, the leaf of the tree. 
Thus, there is only one branch to search: the leaf. So $P(t)$ holds.

\medskip
\emph{Inductive step: t = $\mname{Fork}(b_1, n, b_2)$}. Assume $P(b_1)$ and $P(b_2)$ hold. Prove $P(t)$ holds.

\medskip
According to the properties of binary search trees we know that $t$ follows the binary search property: that all leaves containing
values less than or equal to $n$ are stored in $b_1$, and any leaves with values greater than or equal to $n$ are stored in $b_2$. Thus,
the search operation algorithm can compare the search value $x$ with $n$ and choose which sub-tree to search in.

\medskip
We then have 3 cases.
\begin{itemize}
    \item If $x = n$ the search operation has only had to check one value or one branch of possible
    values so $P(t)$ holds.

    \item  If $x < n$ the search operation can choose the left branch $b_1$ to continue its search. Since $P(b_1)$
    holds we know the search operation will only search one branch in the left sub-tree as well. Together, the search operation will
    only have to search one branch overall of $t$. Therefore, $P(t)$ holds for $x < n$.

    \item Similarly, we can follow the same logic for $x > n$ with the right branch $b_2$
    since $P(b_2)$ holds. Thus, $P(t)$ holds for all cases.
\end{itemize}   

\medskip
We have shown that $P(t)$ holds for all binary trees modeled by $\mname{BinTree}$, thus search operations need only to search one branch
of binary search trees.

\end {document}