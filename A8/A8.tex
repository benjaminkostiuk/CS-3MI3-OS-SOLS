\documentclass[12pt, fleqn]{article}

\usepackage[utf8]{inputenc}
\usepackage{graphicx}
\usepackage{paralist}
\usepackage{booktabs}
\usepackage{hyperref}
\usepackage{amsmath}
\usepackage{amsfonts}
\usepackage{amssymb}
\usepackage{amsthm}
\usepackage{color}
\usepackage[nounderscore]{syntax}
% \usepackage{mdwtab}
\usepackage{mathpartir}
\usepackage{flagderiv}

% \renewcommand{\texttt}[1]{\OldTexttt{\color{teal}{#1}}}
\newcommand{\pnote}[1]{{\langle \text{#1} \rangle}}
\newcommand{\mname}[1]{\mbox{\sf #1}}
\definecolor{mGreen}{rgb}{0,0.6,0}
\definecolor{mGray}{rgb}{0.5,0.5,0.5}
\definecolor{mPurple}{rgb}{0.58,0,0.05}
\definecolor{mGreen2}{rgb}{0.05,0.65,0.05}
\definecolor{mGray2}{rgb}{0.55,0.55,0.55}
\definecolor{mPurple2}{rgb}{0.63,0.05,0.05}
\definecolor{backgroundColour}{rgb}{0.95,0.95,0.92}
\definecolor{backgroundColour2}{rgb}{0.95,0.92,0.95}

\definecolor{t_comment}{rgb}{0.2,1,0.2}
\definecolor{t_mGray}{rgb}{0.5,0.5,0.5}
\definecolor{t_mPurple}{rgb}{0.58,0,0.05}
\definecolor{t_blue}{rgb}{0.4,0.6,0.8}
\definecolor{t_mGreen2}{rgb}{0.05,0.65,0.05}
\definecolor{t_mGray2}{rgb}{0.75,0.75,0.75}
\definecolor{t_mPurple2}{rgb}{0.63,0.05,0.05}
\definecolor{t_bg}{rgb}{0.15,0.15,0.18}

\setlength{\parindent}{0pt}
\setlength {\topmargin} {-.15in}
\oddsidemargin 0mm
\evensidemargin 0mm
\textwidth 160mm
\textheight 200mm

\pagestyle {plain}
\pagenumbering{arabic}

\newcounter{stepnum}

\title{3MI3 Assignment 8}
\author{x}
\date{\today}

\begin{document}


\begin{center}

    {\large \textbf{COMPSCI 3MI3}}\\[8mm]
    {\huge \textbf{Assignment 8}}\\[6mm]
    
\end{center}

\medskip

\section{Solution Set}

\subsection{Q1}

Using the definition of simply typed $\lambda$-Calculus presented in the slides from topic 9,
we define $\lambda^{\mathcal{E}}$ as our external calculus composed of simply typed $\lambda$-Calculus enriched
with the Unity type and term, as well as the $\texttt{;}$ term, evaluation rules E-Seq, E-SeqNext and typing rule T-Seq.
Similarly we define $\lambda^{\mathcal{I}}$ as our internal calculus composed of simply typed $\lambda$-Calculus and the
Unit type and term \emph{only}.

We then define an elboration function $e : \lambda^{\mathcal{E}} \rightarrow \lambda^{\mathcal{I}}$, which translates terms
from $\lambda^{\mathcal{E}}$ to $\lambda^{\mathcal{I}}$. We define $e$ with the following inference rule: 
\begin{center}
    \begin{mathpar}
        \inferrule{
            x \notin FV(t_2') \and e(t_1) = t_1' \and e(t_2) = t_2'
        }
        {
            e(t_1 \: \textrm{;}\: t_2) = 
            (\lambda x : \textrm{Unit}.\:t_2')\:t_1'
        }
    \end{mathpar}
\end{center}
For all other cases where the argument passed to $e$ does not use the sequencing operator (;) we define $e$ to simply return
its argument such that $e(t) = t$ for the term $t$.

\begin{enumerate}[(a)]
    \item We will prove the property $P(t) = t \xrightarrow{\mathcal{E}} t' \implies e(t) \xrightarrow{\mathcal{I}} e(t')$ for all terms
    $t$ in $\lambda^\mathcal{E}$ by structural induction on $t$. We proceed by case analysis on $t$.
    We can group the cases on $t$ into two groups: those for which an inference rule on $e$ exists and those for which a rule does not
    exist. 

    For cases on $t$ in which $e(t) = t$ the proof is trivial.
    By our assumption we know $t \xrightarrow{\mathcal{E}} t'$ which is equivalent to $t \xrightarrow{\mathcal{I}} t'$ by $e(t) = t$.
    Since the only the evaluations rules in $\lambda^{\mathcal{E}}$ that are not in $\lambda^{\mathcal{I}}$ involve sequencing, 
    and $t$ does not contain sequencing, we can say that $t$ also evaluates to $t'$ in our internal calculus. Thus our property holds. 
    This case includes all our base cases on $t$.

    For rest of the cases on $t$, there exists an inference rule for $t$ on $e$. The term must then be of the form $t = t_1\:\textrm{;}\:t_2$
    As part of our induction step we assume our property holds for subterms of $t$, $t_1$ and $t_2$. We will prove our property holds for $t$. 
    From our assumption we know $t \xrightarrow{\mathcal{E}} t'$, thus the evaluation derivation must be one of the following:
    
    \textbf{Case:} E-Seq\\
    If $t$ evaluates to $t'$ by E-Seq then we know $t = t_1\:\textrm{;}\:t_2$, $t' = t_1'\:\textrm{;}\:t_2$ and have the premise
    $t_1 \xrightarrow{\mathcal{E}} t_1'$. By the definition of $e$ we know $e(t) = (\lambda x : \textrm{Unit}.\:t_2'')\:t_1''$ with
    the premises $x \notin FV(t_2')$, $e(t_1) = t_1''$ and $e(t_2) = t_2''$.
    By our induction hypothesis on $t_1$ we know if $t_1 \xrightarrow{\mathcal{E}} t_1'$ then $e(t_1) \xrightarrow{\mathcal{I}} e(t_1')$,
    and by our premise $e(t_1) = t_1''$ we know $e(t_1) \xrightarrow{\mathcal{I}} e(t_1')$ is equivalent to $t_1'' \xrightarrow{\mathcal{I}} e(t_1')$.
    Thus, by E-App2 and the premise $t_1'' \xrightarrow{\mathcal{I}} e(t_1')$ we have
    $(\lambda x : \textrm{Unit}.\:t_2'')\:t_1'' \xrightarrow{\mathcal{I}} (\lambda x : \textrm{Unit}.\:t_2'')\:y$ where $y = e(t_1')$. By our
    inference rule on $e$ with the premises $x \notin FV(t_2')$, $e(t_1') = y$ and $e(t_2) = t_2''$ we have
    $(\lambda x : \textrm{Unit}.\:t_2'')\:y = e(t_1'\:\textrm{;}\:t_2)$. Therefore we have $e(t) \xrightarrow{\mathcal{I}} e(t')$ and our property holds for
    the E-Seq case.

    \textbf{Case:} E-SeqNext\\
    If $t$ evaluates to $t'$ by E-SeqNext then we know $t = \texttt{unit}\:;\:t_2$ and $t' = t_2$. By the definition of $e$
    we know $e(t) = (\lambda x : \textrm{Unit}.\:t_2')\:\:\texttt{unit}$ and have the premises $x \notin FV(t_2')$ and $e(t_2) = t_2'$.
    By E-AppAbs we know $(\lambda x : \textrm{Unit}.\:t_2')\:\:\texttt{unit} \xrightarrow{\mathcal{I}} t_2'$, we throw away the argument $x$ since we know $x \notin FV(t_2')$.
    This is equivalent to $e(t) \xrightarrow{\mathcal{I}} t_2'$, and by premises $e(t') = e(t_2) = t_2'$. Therefore we've shown
    $e(t) \xrightarrow{\mathcal{I}} e(t')$ and our property holds.

    Thus we've shown that our property holds for terms for which there exists an inference rule on $e$ as part of our
    induction step. Thus our induction step holds for all cases.

    Therefore, we've shown that our property $P(t)$ holds for all terms $t$ in $\lambda^{\mathcal{E}}$ by structural 
    induction on $t$.\qed

    \item We will prove the property $P(t) = e(t) \xrightarrow{\mathcal{I}} e(t') \implies t \xrightarrow{\mathcal{E}} t'$ for all terms
    $t$ in $\lambda^\mathcal{E}$ by structural induction on $t$. We proceed by case analysis on $t$.
    We can group the cases on $t$ into two groups: those for which an inference rule on $e$ exists and those for which a rule does not
    exist.

    For cases on $t$ in which $e(t) = t$ the proof is trivial.
    By our assumption we know $e(t) \xrightarrow{\mathcal{I}} e(t')$, which is equivalent to $t \xrightarrow{\mathcal{I}} t'$
    by $e(t) = t$. We know all evaluation rules in $\lambda^{\mathcal{I}}$ are present in $\lambda^\mathcal{E}$, thus we also have 
    $t \xrightarrow{\mathcal{E}} t'$ and our property holds. This case includes all our base cases on $t$.
    
    For rest of the cases on $t$, there exists an inference rule for $t$ on $e$. The term must then be of the form $t = t_1\:\textrm{;}\:t_2$,
    with $e(t) = (\lambda x : \textrm{Unit}.\:t_2')\:t_1'$ and the premises $x \notin FV(t_2')$, $e(t_1) = t_1'$ and $e(t_2) = t_2'$.
    As part of our induction step we assume our property holds for subterms of $t$: $t_1$ and $t_2$. We will prove our property holds for $t$. 
    From our assumption we know $e(t) \xrightarrow{\mathcal{I}} e(t')$, thus the evaluation derivation must be one of the following:

    \textbf{Case:} E-App2\\
    If $e(t)$ evaluates to $e(t')$ by E-App2 then we know $e(t') = (\lambda x: \textrm{Unit}.\:t_2')\:\:t_1''$ and have the premise
    $t_1' \xrightarrow{\mathcal{I}} t_1''$. Setting $e(y) = t_1''$, by the definition of $e$ with the premises $x \notin FV(t_2')$
    and $e(t_2) = t_2'$ we have $t' = y\:\textrm{;}\:t_2$. By our induction hypothesis on $t_1$ with the premises $e(t_1) = t_1'$ and
    $t_1' \xrightarrow{\mathcal{I}} t_1''$ we know $t_1 \xrightarrow{\mathcal{E}} y$. Since $t_1 \xrightarrow{\mathcal{E}} y$, by E-Seq
    we know $t_1\:\textrm{;}\:t_2 \xrightarrow{\mathcal{E}} y\:\textrm{;}\:t_2$, which is $t \xrightarrow{\mathcal{E}} t'$. Thus our
    property holds for the E-App2 case.

    \textbf{Case:} E-AppAbs\\
    If $e(t)$ evaluates to $e(t')$ by E-AppAbs then we know $e(t') = t_2'$ since $x \notin FV(t_2')$ by our premises.
    By our premises $e(t_2) = t_2'$ thus $t' = t_2$. If $e(t)$ is well typed then we know $t_1$ must be a value of type Unit.
    The only value of type Unit is \texttt{unit}. Thus by setting $t_1 = \texttt{unit}$ and using the E-SeqNext evaluation
    rule we have $t \xrightarrow{\mathcal{E}} t'$ and our property holds for the E-AppAbs case.

    Thus we've shown that our property holds for terms for which there exists an inference rule on $e$ as part of our
    induction step. Thus our induction step holds for all cases.

    Therefore, we've shown that our property $P(t)$ holds for all terms $t$ in $\lambda^{\mathcal{E}}$ by structural 
    induction on $t$.\qed

    \item We will prove the property $P(t) = \Gamma \vdash^{\mathcal{E}} t : T \implies \Gamma \vdash^{\mathcal{I}} e(t) : T$ for all terms
    $t$ in $\lambda^\mathcal{E}$ by structural induction on $t$. We proceed by case analysis on $t$.
    We can group the cases on $t$ into two groups: those for which an inference rule on $e$ exists and those for which a rule does not
    exist. 

    For cases on $t$ in which $e(t) = t$ the proof is trivial.
    By our assumption we know $t$ is well-typed as type $T$ in context $\Gamma$ in the external calculus, thus there exists an typing rule for 
    $\Gamma \vdash^{\mathcal{E}} t : T$.
    Since we know there does not exist an inference rule on $e$ for $t$ then $t$ does not contain the sequencing operator and $e(t) = t$.
    Thus we know $\Gamma \vdash^{\mathcal{I}} e(t) : T$ is equivalent to $\Gamma \vdash^{\mathcal{I}} t : T$. Since the only the typing rules in
    $\lambda^{\mathcal{E}}$ that are not in $\lambda^{\mathcal{I}}$ involve sequencing, and $t$ does not contain sequencing, we can say that $t$ is
    also well-typed as type $T$ in our internal calculus. Thus our property holds. This case includes all our base cases on $t$.

    For rest of the cases on $t$, there exists an inference rule for $t$ on $e$. The term must then be of the form $t = t_1\:\textrm{;}\:t_2$
    As part of our induction step we assume our property holds for subterms of $t$: $t_1$ and $t_2$. We will prove our property holds for $t$. 
    From our assumption we know $\Gamma \vdash^{\mathcal{E}} t : T$, thus the typing derivation must be T-Seq.

    If $t$ is typed $T$ in context $\Gamma$ by T-Seq then we know $t = t_1\:\textrm{;}\:t_2$ and have the premises $\Gamma \vdash^{\mathcal{E}}
    t_1 : \textrm{Unit}$ and $\Gamma \vdash^{\mathcal{E}} t_2 : T$. By the definition of $e$ we know $e(t) = (\lambda x: \textrm{Unit}.\:
    t_2')\:\:t_1'$, with premises $x \notin FV(t_2')$, $e(t_1) = t_1'$ and $e(t_2) = t_2'$. By our induction hypothesis on $t_1$ and with the premises
    $\Gamma \vdash^{\mathcal{E}} t_1 : \textrm{Unit}$ and $e(t_1) = t_1'$, we know $\Gamma \vdash^{\mathcal{I}} t_1' : \textrm{Unit}$.
    Similarly, by our induction hypothesis on $t_2$ with the premises $\Gamma \vdash^{\mathcal{E}} t_2 : T$ and $e(t_2) = t_2'$ we know 
    $\Gamma \vdash^{\mathcal{T}} t_2' : T$.\\
    Since $x \notin FV(t_2')$ we can assume there exists no typing information about $x$, which
    implies $x \notin dom(\Gamma)$. Thus, we have $\Gamma \vdash^{\mathcal{T}} t_2' : T$ and $x \notin dom(\Gamma)$, by the weakening lemma we have $\Gamma, x: \textrm{Unit} \vdash^{\mathcal{T}} t_2' : T$.
    Therefore by T-Abs we have $\Gamma \vdash^{\mathcal{I}} (\lambda x: \textrm{Unit}.\:t_2') : \textrm{Unit} \Rightarrow T$.\\
    Since we know $\Gamma \vdash^{\mathcal{I}} t_1' : \textrm{Unit}$ and $\Gamma \vdash^{\mathcal{I}} (\lambda x: \textrm{Unit}.\:t_2')
    : \textrm{Unit} \Rightarrow T$, then by T-App we can conclude $\Gamma \vdash^{\mathcal{I}} (\lambda x: \textrm{Unit}.\:
    t_2')\:\:t_1' : T$. Thus $\Gamma \vdash^{\mathcal{I}} e(t) : T$ and our property holds.

    Thus we've shown that our property holds for terms for which there exists an inference rule on $e$ as part of our
    induction step. Thus our induction step holds for all cases.

    Therefore, we've shown that our property $P(t)$ holds for all terms $t$ in $\lambda^{\mathcal{E}}$ by structural 
    induction on $t$.\qed

    \item We will prove the property $P(t) = \Gamma \vdash^{\mathcal{I}} e(t) : T \implies \Gamma \vdash^{\mathcal{E}} t : T$ for all terms
    $t$ in $\lambda^\mathcal{E}$ by structural induction on $t$. We proceed by case analysis on $t$.
    We can group the cases on $t$ into two groups: those for which an inference rule on $e$ exists and those for which a rule does not
    exist.

    For cases on $t$ in which $e(t) = t$ the proof is trivial.
    By our assumption we know $\Gamma \vdash^{\mathcal{I}} e(t) : T$, which is equivalent to $\Gamma \vdash^{\mathcal{I}} t : T$
    by $e(t) = t$. We know all typing rules in $\lambda^{\mathcal{I}}$ are present in $\lambda^\mathcal{E}$, thus we also have 
    $\Gamma \vdash^{\mathcal{E}} t : T$ and our property holds. This case includes all our base cases on $t$.

    For rest of the cases on $t$, there exists an inference rule for $t$ on $e$. The term must then be of the form $t = t_1\:\textrm{;}\:t_2$,
    with $e(t) = (\lambda x: \textrm{Unit}.\:t_2')\:\:t_1'$ and the premises $x \notin FV(t_2')$, $e(t_1) = t_1'$ and $e(t_2) = t_2'$.
    As part of our induction step we assume our property holds for subterms of $t$: $t_1$ and $t_2$. We will prove our property holds for $t$. 
    From our assumption we know $\Gamma \vdash^{\mathcal{I}} e(t) : T$, thus the typing derivation must be T-App.

    If $e(t)$ is typed $T$ in context $\Gamma$ by T-App then we know $T = T_2$ and have the premises $\Gamma \vdash^{\mathcal{I}}
    (\lambda x: \textrm{Unit}.\:t_2') : T_1 \Rightarrow T_2$ and $\Gamma \vdash^{\mathcal{I}} t_1' : T_1$. Since we have
    $\Gamma \vdash^{\mathcal{I}} (\lambda x: \textrm{Unit}.\:t_2') : T_1 \Rightarrow T_2$, by T-Abs we have the premise
    $\Gamma, x: T_1 \vdash^{\mathcal{I}} t_2' : T_2$, with $T_1 = \textrm{Unit}$. Since $x \notin FV(t_2')$ we can discard $x$
    from our typing context without losing information about $t_2'$ such that $\Gamma \vdash^{\mathcal{I}} t_2' : T_2$, and since $T_1 = \textrm{Unit}$ we have $\Gamma \vdash^{\mathcal{I}} t_1' : \textrm{Unit}$.
    By applying these premises to our induction hypothesis for $t_1$ and $t_2$ since we know $e(t_1) = t_1'$ and $e(t_2) = t_2'$ we can
    show $\Gamma \vdash^{\mathcal{E}} t_1 : \textrm{Unit}$ and $\Gamma \vdash^{\mathcal{E}} t_2 : T_2$. Therefore, by
    T-Seq we know $\Gamma \vdash^{\mathcal{E}} (t_1\:\textrm{;}\:t_2) : T_2$ which is $\Gamma \vdash^{\mathcal{E}} t : T$.
    Thus our property holds.

    Thus we've shown that our property holds for terms for which there exists an inference rule on $e$ as part of our
    induction step. Thus our induction step holds for all cases.

    Therefore, we've shown that our property $P(t)$ holds for all terms $t$ in $\lambda^{\mathcal{E}}$ by structural 
    induction on $t$.\qed

\end{enumerate}


\end {document}