\documentclass[12pt, fleqn]{article}

\usepackage[utf8]{inputenc}
\usepackage{graphicx}
\usepackage{paralist}
\usepackage{booktabs}
\usepackage{hyperref}
\usepackage{amsmath}
\usepackage{amsfonts}
\usepackage{amssymb}
\usepackage{amsthm}
\usepackage{color}
\usepackage[nounderscore]{syntax}
\usepackage{mdwtab}
\usepackage{mathpartir}

% \renewcommand{\texttt}[1]{\OldTexttt{\color{teal}{#1}}}
\newcommand{\pnote}[1]{{\langle \text{#1} \rangle}}
\newcommand{\mname}[1]{\mbox{\sf #1}}
\definecolor{mGreen}{rgb}{0,0.6,0}
\definecolor{mGray}{rgb}{0.5,0.5,0.5}
\definecolor{mPurple}{rgb}{0.58,0,0.05}
\definecolor{mGreen2}{rgb}{0.05,0.65,0.05}
\definecolor{mGray2}{rgb}{0.55,0.55,0.55}
\definecolor{mPurple2}{rgb}{0.63,0.05,0.05}
\definecolor{backgroundColour}{rgb}{0.95,0.95,0.92}
\definecolor{backgroundColour2}{rgb}{0.95,0.92,0.95}

\definecolor{t_comment}{rgb}{0.2,1,0.2}
\definecolor{t_mGray}{rgb}{0.5,0.5,0.5}
\definecolor{t_mPurple}{rgb}{0.58,0,0.05}
\definecolor{t_blue}{rgb}{0.4,0.6,0.8}
\definecolor{t_mGreen2}{rgb}{0.05,0.65,0.05}
\definecolor{t_mGray2}{rgb}{0.75,0.75,0.75}
\definecolor{t_mPurple2}{rgb}{0.63,0.05,0.05}
\definecolor{t_bg}{rgb}{0.15,0.15,0.18}

\setlength{\parindent}{0pt}
\setlength {\topmargin} {-.15in}
\oddsidemargin 0mm
\evensidemargin 0mm
\textwidth 160mm
\textheight 200mm

\pagestyle {plain}
\pagenumbering{arabic}

\newcounter{stepnum}

\title{3MI3 Assignment 6 Solution}
\author{x}
\date{\today}

\begin{document}


\begin{center}

    {\large \textbf{COMPSCI 3MI3}}\\[8mm]
    {\huge \textbf{Assignment 6}}\\[6mm]
\end{center}

\medskip

\section{Solution Set}

\subsection{Q1}

We assume the syntax and semantics for TAE as defined in the slides for topic 4 and 7.
We define the syntax for a \emph{logical and} operator, and \emph{arithmetic addition} operator with the
following syntax in EBNF:

\begin{verbatim}
    t ::= ...
        | and t t
        | plus t t

    v ::= ...
    nv ::= ...
    T ::= ...
\end{verbatim}
Where \verb|t| defines terms, \verb|v| values, \verb|nv| numerical values and \verb|T| the set of types for TAE as given in the slides.
We add our logical and operator \verb|and|, as well as our arithmetic addition operator \verb|plus| to the definition
of \verb|t|.

We use the same set of types \verb|T| as defined in slide set 7 and assume any previously defined typing rules hold.

\begin{enumerate}[(a)]
    \item We define the following typing rules for our logical and operator \verb|and|:
\begin{center}
    \begin{equation}
        \tag{T-And}
        \inferrule{
            t_1 : \texttt{Bool}
            \:\:\and
            t_2 : \texttt{Bool}
        }
        {
            \texttt{and} \:\: t_1 \:\: t_2 : \texttt{Bool}
        }
    \end{equation}
\end{center}

\medskip
    \item We define the following typing rules for our arithmetic addition operator \verb|plus|:
\begin{center}
    \begin{equation}
        \tag{T-Plus}
        \inferrule{
            t_1 : \texttt{Nat}
            \:\:\and
            t_2 : \texttt{Nat}
        }
        {
            \texttt{plus} \:\: t_1 \:\: t_2 : \texttt{Nat}
        }
    \end{equation}
\end{center}
\end{enumerate}

\subsection{Q2}

We will prove the property that every subterm of a well-typed term is also well-typed, for all
terms in TAE. Formally, if $t_1$ is a subterm of term $t : \alpha$, then we have $t_1 : \beta$ with
$\alpha, \beta \in \texttt{T}$, $\texttt{T}$ the set of types in TAE.
We will prove this property holds by induction on the typing derivation $t : \alpha$. We consider the typing
derivations defined in the slides for topic 7.

\medskip
\emph{Base case.} The base case denotes typing derivations on terms for which there does not exists any sub-typing
derivations. The possible typing derivations are of the following:

\medskip
\textbf{Case}: T-True, T-False, T-Zero\\
If T-True is the final typing derivation we know $t = \texttt{true}$ and $\texttt{T} = \texttt{Bool}$. Since
\verb|true| does not have any subterms, the left hand side of our implication is false, thus our property is
vacuously true. We can apply the same argument for T-False and T-Zero.

Thus, our base case holds.

\medskip
\emph{Induction step.} We assume that for all typing sub-derivations the above property holds.
We will prove that for the final typing derivation our property holds. The typing derivation must be one of the
following:

\medskip
\textbf{Case}: T-If\\
If T-If was the last typing derivation we know $t = \texttt{if}\:t_1\:\texttt{then}\:t_2\:\texttt{else}\:t_3 : \texttt{T}$,
and have the premises that $t_1 : \texttt{Bool}$, $t_2 : \texttt{T}$ and $t_3 : \texttt{T}$. By our induction hypothesis we know
that if $t_1$ is well typed then all of its subterms are well-typed. Since $t_1 : \texttt{Bool}$ we have that all subterms of $t_1$
are well-typed. We can apply the same argument for $t_2$ and $t_3$. Thus, all subterms of $t_1$, $t_2$ and $t_3$ are well-typed.
Therefore for our term $t : \texttt{T}$ we've shown $t_1$, $t_2$ and $t_3$ are well-typed by our premise, and all subterms of $t_1$, $t_2$
and $t_3$ are well-typed, thus all subterms of $t$ must be well-typed. 
Therefore our property holds for the T-If case. 

\medskip
\textbf{Case}: T-Pred\\
If T-Pred was the last typing derivation we know $t = \texttt{pred}~t_1 : \texttt{Nat}$, and have the premise
that $t_1 : \texttt{Nat}$. By our induction hypothesis and since we know that $t_1$ is well-typed because of our premise,
we know that all subterms of $t_1$ are also well-typed. Thus, for our term $t : \texttt{Nat}$ we've show that
$t_1$ is well-typed and all subterms of $t_1$ are well-typed, therefore all all subterms of $t$ are well typed.
Therefore our property holds for the T-Pred case.

\medskip
\textbf{Case}: T-Succ\\
We can prove the T-Succ case similar to T-Pred. If T-Succ is the last typing derivation we know $t = \texttt{succ}~t_1
 : \texttt{Nat}$, and we have the premise that $t_1 : \texttt{Nat}$. By supplying our induction hypothesis with
our premise that $t_1$ is well-typed we have that all subterms of $t_1$ are also well-typed. 
Thus, for our term $t : \texttt{Nat}$ we've show that $t_1$ is well-typed and all subterms of $t_1$ are 
well-typed, therefore all all subterms of $t$ are well typed.
Therefore our property holds for the T-Succ case.

\medskip
\textbf{Case}: T-IsZero\\
We can prove the T-IsZero case similar to T-Pred. If T-IsZero was the last typing derivation we know $t = \texttt{iszero}~t_1
 : \texttt{Bool}$, and have the premise that $t_1 : \texttt{Nat}$. By our induction hypothesis and since we 
know that $t_1$ is well-typed because of our premise, we know that all subterms of $t_1$ are also well-typed. 
Thus, for our term $t : \texttt{Bool}$ we've show that $t_1$ is well-typed and all subterms of $t_1$ are 
well-typed, therefore all all subterms of $t$ are well typed.
Therefore our property holds for the T-IsZero case.

\medskip
We have shown that for all possible typing derivations over $t$ over property holds. Therefore our induction step holds.

\medskip
Thus, we've shown that every subterm of a well-typed term is also well-typed, for all terms in TAE.\qed

\subsection{Q3}
We will prove the property of preservation holds for all terms in TAE. Formally, for terms $t, t'$ and type
\verb|T| in TAE the following holds:
$$t : \texttt{T} \land t \rightarrow t' \implies t' : \texttt{T}$$
We will prove the property of preservation holds by induction on the derivation $t \rightarrow t'$.

\medskip
\emph{Base case.} The base case denotes the final derivation where $t \rightarrow t'$ will yield a value instead
of a term evaluatable by a sub-derivation. The possible derivations that yield a value are of the following:

\medskip
\textbf{Case}: E-IfTrue, E-IfFalse\\
If E-IfTrue was the last derivation and yields a value, we know $t = \texttt{if}\:\texttt{true}\:\texttt{then}\:
t_2\:\texttt{else}\:t_3$ and $t' = t_2$ with $t_2$ being some value. The only typing rule we can apply to $t$ is
T-If, thus we know $t : \texttt{T}$, $t_2 : \texttt{T}$ and $t_3 : \texttt{T}$ with $\texttt{T}$ a type in TAE. Since
$t : \texttt{T}$ and $t_2 = t' : \texttt{T}$ for the derivation $t \rightarrow t'$ of E-IfTrue our, property holds.
We can prove E-IfFalse using the same argument, swapping $t' = t_3$. Thus, our property of preservation holds for E-IfTrue
and E-IfFalse.

\medskip
\textbf{Case}: E-PredZero\\
If E-PredZero was the last derivation we immediately know $t = \texttt{pred}\:0$ and $t' = 0$, with $t'$ a value.
By T-Pred we immediately know $t : \texttt{Nat}$, and by T-Zero we know $t' : \texttt{Nat}$. Thus, our property
of preservation holds for the E-PredZero case.

\medskip
\textbf{Case}: E-PredSucc\\
If E-PredSucc was the last derivation we know  $t = \texttt{pred}\:(\texttt{succ}\:t_2)$ and $t' = t_2$,
where $t_2$ is a numerical value. The only typing rule we can apply to $t$ is T-Pred, thus we know $t : \texttt{Nat}$
and $\texttt{succ}\:t_2 : \texttt{Nat}$. By the inversion lemma we know $\texttt{succ}\:t_2 : R \implies R = \texttt{Nat} \land
t_2 : \texttt{Nat}$. Therefore we know $t : \texttt{Nat}$ and $t_2 = t : \texttt{Nat}$ for the derivation $t \rightarrow t'$ of
E-PredSucc. Thus, our property holds.

\medskip
\textbf{Case}: E-IsZeroZero\\
If E-IsZeroZero was the last derivation we immediately know $t = \texttt{iszero}\:0$ and $t' = \texttt{true}$, with
$t'$ a value. We can only apply T-IsZero to $t$, thus we know $t : \texttt{Bool}$, and by T-True we know $t' : \texttt{Bool}$
as well. Thus, our property of preservation holds for the E-IsZeroZero case.

\medskip
\textbf{Case}: E-IsZeroSucc\\
If E-IsZeroSucc was the last derivation we know $t = \texttt{iszero}\:(\texttt{succ}\:t_2)$ with $t_2$ a numerical value,
and we know $t' = \texttt{false}$. The only typing rule we can apply to $t$ is T-IsZero, thus we know $t : \texttt{Bool}$, 
and by T-False we know $t' : \texttt{Bool}$ as well. Therefore we know $t : \texttt{Bool}$ and $t' : \texttt{Bool}$ for the
derivation $t \rightarrow t'$ of E-IsZeroSucc. Thus, our property holds.

\medskip
Thus, our base case holds.

\medskip
\emph{Induction step.} We assume that for all sub-derivations of $t \rightarrow t'$ the property of preservation
holds. We will prove that for all possible derivations $t \rightarrow t'$ the property holds. The derivation $t
\rightarrow t'$ must be one of the following:

\medskip
\textbf{Case}: E-IfTrue, E-IfFalse\\
We can prove the cases of E-IfTrue and E-IfFalse in exactly the same manner as for the base case, this time assuming that
the derivation $t \rightarrow t'$ for E-IfTrue and E-IfFalse do not yield a value, thus $t'$ is not a value.
Thus, our property holds for the case E-IfTrue and E-IfFalse

\medskip
\textbf{Case}: E-If\\
If E-IF was the last derivation, we know $t = \texttt{if}\:t_1\:\texttt{then}\:t_2\:\texttt{else}\:t_3$, 
$t' = \texttt{if}\:t_1'\:\texttt{then}\:t_2\:\texttt{else}\:t_3$ and we have the premise that $t_1 \rightarrow t_1'$.
The only typing rule we can apply to $t$ is T-If. From T-If we know $t_1 : \texttt{Bool}$, $t_2 : \texttt{T}$,
$t_3 : \texttt{T}$ and $t : \texttt{T}$ with $\texttt{T}$ a type in TAE. By our induction hypothesis we have
$t_1 : \texttt{T} \land t_1 \rightarrow t_1' \implies t_1' : \texttt{T}$. Since we have the premises $t_1 : \texttt{Bool}$
and $t_1 \rightarrow t_1'$ we can conclude $t_1' : \texttt{Bool}$. Since $t' = \texttt{if}\:t_1'\:\texttt{then}
\:t_2\:\texttt{else}\:t_3$, with $t_1' : \texttt{Bool}$, $t_2 : \texttt{T}$ and $t_3 : \texttt{T}$, by T-If we can
show $t' : \texttt{T}$.
Thus, we've shown the property of preservation holds if the last derivation is E-If.


\medskip
\textbf{Case}: E-Succ\\
If E-Succ was the last derivation, we know $t = \texttt{succ}\:t_1$, $t' = \texttt{succ}\:t_1'$ and have the
premise that $t_1 \rightarrow t_1'$. Since $t = \texttt{succ}\:t_1$, we can only apply the typing derivation
T-Succ. From T-Succ we know $t : \texttt{Nat}$ and have the premise $t_1 : \texttt{Nat}$. By our induction
hypothesis we have $t_1 : \texttt{T} \land t_1 \rightarrow t_1' \implies t_1' : \texttt{T}$. Since we have the premises
$t_1 : \texttt{Nat}$ and $t_1 \rightarrow t_1'$ we can conclude $t_1' : \texttt{Nat}$. We now know $t_1 : \texttt{Nat}$
and $t_1' = \texttt{succ}\:t_1'$, then by T-Succ we can show $t' : \texttt{Nat}$.
Thus, we've shown the property of preservation holds if the last derivation is E-Succ.

\medskip
\textbf{Case}: E-Pred\\
We proceed in the same manner as with E-Succ. If 
If E-Pred was the last derivation, we know $t = \texttt{pred}\:t_1$, $t' = \texttt{pred}\:t_1'$ and have the
premise that $t_1 \rightarrow t_1'$. Since $t = \texttt{succ}\:t_1$, we can only apply the typing derivation
T-Pred. From T-Pred we know $t : \texttt{Nat}$ and have the premise $t_1 : \texttt{Nat}$. By our induction
hypothesis we have $t_1 : \texttt{T} \land t_1 \rightarrow t_1' \implies t_1' : \texttt{T}$. Since we have the premises
$t_1 : \texttt{Nat}$ and $t_1 \rightarrow t_1'$ we can conclude $t_1' : \texttt{Nat}$. We now know $t_1 : \texttt{Nat}$
and $t_1' = \texttt{pred}\:t_1'$, then by T-Pred we can show $t' : \texttt{Nat}$.
Thus, we've shown the property of preservation holds if the last derivation is E-Pred.

\medskip
\textbf{Case}: E-IsZero\\
We can prove the E-IsZero case in similar fashion to the E-Succ case.
If E-IsZero was the last derivation, we know $t = \texttt{iszero}\:t_1$, $t' = \texttt{iszero}\:t_1'$ and have the
premise that $t_1 \rightarrow t_1'$. Since $t = \texttt{iszero}\:t_1$, we can only apply the typing derivation
T-IsZero. From T-IsZero we know $t : \texttt{Bool}$ and have the premise $t_1 : \texttt{Nat}$. By our induction
hypothesis we have $t_1 : \texttt{T} \land t_1 \rightarrow t_1' \implies t_1' : \texttt{T}$. Since we have the premises
$t_1 : \texttt{Nat}$ and $t_1 \rightarrow t_1'$ we can conclude $t_1' : \texttt{Nat}$. We now know $t_1 : \texttt{Nat}$
and $t_1' = \texttt{pred}\:t_1'$, then by T-IsZero we can show $t' : \texttt{Bool}$.
Thus, we've shown the property of preservation holds if the last derivation is E-IsZero.

\medskip
We've shown our property holds over all cases. Therefore our induction step holds.\\
We have shown that for all possible derivations $t \rightarrow t'$ our property of preservation holds, thus
we've shown the property of preservation holds for all terms in TAE.\qed

\end {document}